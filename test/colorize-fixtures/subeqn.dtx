% \iffalse meta-comment
%
% Copyright 1999-2024 Johannes L. Braams.  All rights reserved.
%
% This file is part of the subeqn package.
% ----------------------------------------
%
% It may be distributed and/or modified under the
% conditions of the LaTeX Project Public License, either version 1.3c
% of this license or (at your option) any later version.
% The latest version of this license is in
%   http://www.latex-project.org/lppl.txt
% and version 1.3c or later is part of all distributions of LaTeX
% version 2008 or later.
%
% This work has the LPPL maintenance status "maintained".
%
% The Current Maintainer of this work is Johannes Braams.
%
% The list of all files belonging to the supertabular package is
% given in the file `MANIFEST.
%
% The list of derived (unpacked) files belonging to the distribution
% and covered by LPPL is defined by the unpacking scripts (with
% extension .ins) which are part of the distribution.
% \fi
%
% \CheckSum{84}
% \iffalse
% Copyright (C) 1999-2024
%     Donald Arsenau at reg.triumf.ca,
%     Johannes Braams texniek at texniek.nl
%
%<*dtx>
\ProvidesFile{subeqn.dtx}
%</dtx>
%<package>\NeedsTeXFormat{LaTeX2e}[1998/06/01]
%<package>\ProvidesPackage{subeqn}
%<sample>\ProvidesFile{subeqn-sample.tex}
%<driver>\ProvidesFile{subeqn.drv}
%\ProvidesFile{subeqn.dtx}
              [2024/07/21 v2.0c subnumbering of equations]
%<*driver>
\documentclass{ltxdoc}
\begin{document}
\pagestyle{myheadings}
\providecommand{\Lenv}[1]{\textsf{#1}}
\providecommand{\Lopt}[1]{\textsf{#1}}
\providecommand{\pkg}[1]{\texttt{#1}}
\providecommand{\file}[1]{\texttt{#1}}
\DocInput{subeqn.dtx}
\end{document}
%</driver>
% \fi
%
%  \GetFileInfo{subeqn.dtx}
%  \title{Subnumbering of equations\thanks{This file
%        has version number \fileversion, last
%        revised \filedate.}}
%  \author{Donald Arsenau \and Johannes Braams}
%  \date{\filedate}
%  \maketitle
%
% \markboth
%      {subeqn package version \fileversion\space as of \filedate}
%      {subeqn package version \fileversion\space as of \filedate}
%
%  \section{Introduction}
%
%    Sometimes it is necessary to be able to refer to subexpressions
%    of an equation. In order to do that these subexpressions should
%    be numbered. In standard \LaTeX\ there is no provision for
%    this. To solve this problem Stephen Gildea once wrote
%    \file{subeqn.sty} for \LaTeX$\:$2.09; Donald Arsenau rewrote the
%    macros and Johannes Braams made them available for \LaTeXe.
%
%    Note that this package is \emph{not} compatible with the package
%    \pkg{subeqnarray}, written by Johannes Braams.
%
%    This package can be used together with the \LaTeX\ options
%    \Lopt{leqno} and \Lopt{fleqn}.
%
%  \section{Available environments}
%
% \DescribeEnv{subeqations} Inside the \Lenv{subeqations} environment
%    \LaTeX's equation environments such as \Lenv{equation} and
%    \Lenv{eqnarray} are numbered as subexpressions. At the same time
%    the number of the (main) equation is kept the same.
%
% \DescribeEnv{subeqnarray} |\begin{subeqnarray}| works like
%    |\begin{subequations}||\begin{eqnarray}|, but saves typing.  A
%    |\label| command given at the very beginning of the first entry
%    defines a |label| for the overall equation number, as if you had
%    typed |\begin{subequations}||\label{xxx}||\begin{eqnarray}|.
%
%  \section{Available commands}
%
% \DescribeMacro{\thesubequation} The command |\thesubequation|
%    controls the labelling of the subexpressions of an equation. You
%    can change the labelling by redefining this command, but the
%    names of the counters may be confusing: The sub-number is given
%    by counter \texttt{equation}, while the overall equation number
%    is given by \texttt{mainequation}.
%
%    There are two ways to reference the overall equation number:
%    through its value, as in |\Roman{mainequation}|, or through
%    |\themainequation|, which gives the text of the normal
%    |\theequation|. Refer to the local sub-number through the value
%    of the \texttt{equation} counter, as in |\alph{equation}|.
%    The default numbering is like 13c, given by:
%\begin{verbatim}
%\newcommand*{\thesubequation}{\themainequation\alph{equation}}
%\end{verbatim}
%
%    Some alternatives:\\
%    A number such as 13.C is achieved by
%\begin{verbatim}
%    \newcommand*{\thesubequation}{\themainequation.\Alph{equation}}
%\end{verbatim}
%    A number such as 13-iii is achieved by
%\begin{verbatim}
%    \newcommand*{\thesubequation}{\themainequation-\roman{equation}}
%    \newcommand*{\thesubequation}{\themainequation.\Alph{equation}}
%\end{verbatim}
%    When the document class which is used has declared  
%\begin{verbatim}
%    \renewcommand{\@eqnnum}{\theequation} 
%    \renewcommand{\theequation}{(\arabic{equation})}
%\end{verbatim}
% which puts parentheses around \emph{all} equation numbers, including
% those produced by the |\ref| command, you can use:
%\begin{verbatim}
%\newcommand*{\thesubequation}{(\arabic{mainequation}\alph{equation})}
%\end{verbatim}
%
%  \StopEventually{}
%
%  \section{The implementation}
%
%    \begin{macrocode}
%<*package>
%    \end{macrocode}
%
%  \begin{environment}{subeqations}
%    Within the \Lenv{subequations} the equation numbers consist of
%    two parts. The first part is a representation of the current value
%    of the \texttt{equation} counter when the environment is entered,
%    ie the number of the equation; the second part indicates the
%    number of the subexpression of the equation.
%    \begin{macrocode}
\newenvironment{subequations}{%
%    \end{macrocode}
%    First we update the \texttt{equation} counter,
%    \begin{macrocode}
  \refstepcounter{equation}%
%    \end{macrocode}
%    then we save its current value in |\c@mainequation| and define
%    |\themainequation| to be the current representation of the
%    \texttt{equation} counter.
%    \begin{macrocode}
  \mathchardef\c@mainequation\c@equation
  \protected@edef\themainequation{\theequation}%
%    \end{macrocode}
%    Then we change the representation of the \texttt{equation}
%    counter to represent the subexpression number. Finally we set the
%    \texttt{equation} counter to zero as we use it for counting the
%    subexpressions. 
%    \begin{macrocode}
  \let\theequation\thesubequation
  \global\c@equation\z@
  }{%
%    \end{macrocode}
%    When the environment is finished we restore the value of the
%    \texttt{equation} counter.
%    \begin{macrocode}
  \global\c@equation\c@mainequation
  \global\@ignoretrue
  }
%    \end{macrocode}
%  \end{environment}
%
%  \begin{macro}{\thesubequation}
%    By default the subexpressions will be numbered with lower case
%    letters. The representation of the \texttt{equation} counter also
%    includes the saved value of the \texttt{equation} counter. This
%    can be changed by redefining this command.
%    \begin{macrocode}
\newcommand{\thesubequation}{\themainequation\alph{equation}}
%    \end{macrocode}
%  \end{macro}
%
%  \begin{environment}{subeqnarray}
%    
%    \begin{macrocode}
\newenvironment{subeqnarray}{%
  \subequations
  \@ifnextchar\label{\@lab@subeqnarray}{\eqnarray}
  }{%
  \endeqnarray\endsubequations
  }
%    \end{macrocode}
%  \end{environment}
%
%  \begin{macro}{\@lab@subeqnarray}
%    This macro picks up the |\label| command and its argument and
%    re-inserts it \emph{before} starting the \Lenv{eqnarray}
%    environment. 
%    \begin{macrocode}
\newcommand*{\@lab@subeqnarray}[2]{#1{#2}\eqnarray}
%    \end{macrocode}
%  \end{macro}
%
%    \begin{macrocode}
%</package>
%    \end{macrocode}
%
% \section{An example of the use of this package}
%
%    When you run the following document through \LaTeX\ you will see
%    the differene between the \texttt{subeqnarray} and
%    \texttt{eqnarray} environments.
%    \begin{macrocode}
%<*sample>
\documentclass{article}
\usepackage{subeqn}

\begin{document}
\title{Sample sub-equations}
\author{Johannes L. Braams}
\date{\today}
\maketitle

\noindent
This is an example of the use of the \texttt{subeqations}
package. First we have a normal \textsf{equation} environment.
\begin{equation}
  \label{a}
  a^2 + b^2 = c^2
\end{equation}
Now we start sub-numbering.
\begin{subequations}
  \label{b}
  \begin{equation}
    \label{b1}
    d^2 + e^2 = f^2
  \end{equation}
  We can refer to equation~\ref{a}, \ref{b} and~\ref{b1}.
  \begin{equation}
    \label{b2}
    g^2 + h^2 = i^2
  \end{equation}
  This was equation~\ref{b2}.
  \begin{eqnarray}
    \label{c}
    x &=& y+z\label{c1}\\
    u &=& v+w\label{c2}
  \end{eqnarray}
  This was expression~\ref{c}, consisting of parts~\ref{c1}
  and~\ref{c2}. 
\end{subequations}

\noindent
Now lets start a \textsf{subeqnarray} environment.
\begin{subeqnarray}
  \label{d}
  x &=& y+z\label{d1}\\
  u &=& v+w\label{d2}
\end{subeqnarray}
This was equation~\ref{d}, with parts~\ref{d1} and~\ref{d2}.  
\end{document}
%</sample>
%    \end{macrocode}
%
% \Finale
\endinput
